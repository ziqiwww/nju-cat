\documentclass[UTF8, 10pt, twocolumn]{article}

\usepackage{ctex} % 中文支持
\usepackage{titlesec} % 设置标题格式
\usepackage{titling} % 设置作者格式
\usepackage{fancyhdr} % 设置页眉页脚
\usepackage{lipsum} % 生成虚拟文本,可删除
\usepackage{hyperref}
\usepackage{cite} % 参考文献
\usepackage{geometry}
\usepackage{graphicx}
\usepackage{booktabs}
\usepackage{xcolor,colortbl}

% 设置页边距
\geometry{a4paper, left=2.5cm, right=2.5cm, top=2.5cm, bottom=2.5cm}
% 设置页眉页脚
% \pagestyle{fancy}
% \fancyhf{} % 清空当前设置
% \rhead{\thepage} % 页眉右侧显示页码
% \lhead{\theauthor} % 页眉左侧显示作者
% \renewcommand{\headrulewidth}{0pt} % 页眉横线宽度
% \renewcommand{\footrulewidth}{0pt} % 页脚横线宽度

\usepackage{listings}
\lstset{language=c++}%这条命令可以让LaTeX排版时将c++关键字突出显示
\lstset{
	breaklines,%这条命令可以让LaTeX自动将长的代码行换行排版
	basicstyle=\footnotesize\ttfamily, % Standardschrift
	backgroundcolor=\color[rgb]{0.95,0.95,0.95},
	keywordstyle=\color{blue},
	commentstyle=\color{cyan},
	tabsize=4,numbers=left,
	numberstyle=\tiny,
	frame=single,
	%numbers=left, % Ort der Zeilennummern
	numberstyle=\tiny, % Stil der Zeilennummern
	%stepnumber=2, % Abstand zwischen den Zeilennummern
	numbersep=5pt, % Abstand der Nummern zum Text
	tabsize=2, % Groesse von Tabs
	extendedchars=false, %
	breaklines=true, % Zeilen werden Umgebrochen
	keywordstyle=\color{red},%这一条命令可以解决代码跨页时, 章节标题, 页眉等汉字不显示的问题
	stringstyle=\color{white}\ttfamily, % Farbe der String
	showspaces=false, % Leerzeichen anzeigen ?
	showtabs=false, % Tabs anzeigen ?
	xleftmargin=17pt,
	framexleftmargin=17pt,
	framexrightmargin=5pt,
	framexbottommargin=4pt,
	%backgroundcolor=\color{lightgray},
	showstringspaces=false % Leerzeichen in Strings anzeigen ?
}
\renewcommand{\lstlistingname}{CODE}
\renewcommand{\tablename}{{\CJKfamily{kai}表}}

\title{并发算法与理论——并发SkipList设计与思考}
\author{王紫萁\thanks{邮编:210046,系楼412, Github:\href{https://github.com/ziqiwww}{ziqiwww}} \quad 学号:201220154 \\
\href{mailto:ziqiwang@smail.nju.edu.cn}{ziqiwang@smail.nju.edu.cn} \\
\href{cs.nju.edu.cn}{南京大学计算机科学与技术系}
}
% email
\date{ \today}

\begin{document}


\maketitle
\begin{abstract}
  作为《并发算法与理论》的课程设计,我们选取了\cite[Herlihy]{prencipe_simple_2007}的经典并发SkipList算法进行复现,并对论文中的概念与算法进行了分析与思考。
  最后,讨论模糊搜索HNSW算法\cite[Malkov]{malkov_efficient_2020}是否能够采用本文的思想。
\end{abstract}

\section{引言}
Skip List是一种基于链表的数据结构,其在插入、删除、查找等操作上的时间复杂度均为$O(\log n)$,并且相比于红黑树等其他数据结构,SkipList 的实现更加简单,且在并发环境下具有较好的性能。
Skip List的并发算法在原论文截止时的相关工作有\cite[Pugh]{pugh_skip_1990}的算法,该算法由于使用了反转指针因此实现非常复杂;Lea, D.在前人工作上进行了改进并把算法应用到了Java的并发库中,该算法的性能
达到了sota,缺点时实现复杂,并且在某些特定的插入与删除交错执行的情况下,跳表的不变性(invariant)会暂时或永久被违背。

Herlihy提出了一种基于锁的算法,该算法特点是:乐观锁(optimistic lock),延迟删除(lazy remove)。论文对该算法的实现非常详细,并且使用Java实现并比较了该算法与Lea的算法的性能,结果表明该算法的性能
和Lea的算法相当,但是实现更加简单,有利于并发环境下的调试与维护。

本报告尝试使用c++复现了朴素与并发跳表算法,并在Add,Remove,Contains三个操作总共200,000条指令上进行了压力测试,采用不同线程进行性能对比,在实验部分会给出具体的数据与分析。

\section{背景与相关工作}
\subsection{跳表}
跳表是一种基于链表的线性数据结构,也是一种典型的分层索引算法。在插入删除的平均时间复杂度为$O(\log n)$。核心思想为:使用空间换取时间,一个元素在跳表中可能出现在多个层级,每个层级都是一个有序链表,最底层的链表包含所有元素,
查找时从最上层开始查找,如果当前层级的下一个元素大于目标元素,则下降到下一层级继续查找,直到找到目标元素或者到达最底层。对于每个节点的最高层数通过参数$P$来控制,$P$越大,跳表的层数越高,查找的效率越高,但是空间开销也越大。
因此选择合适的$P$是也一个开放问题。

\begin{figure*}[ht]
  \begin{lstlisting}[language=c++]
  template<typename T>
  auto ConSkipList<T>::FindNode(
    T key, 
    std::shared_ptr<ConSkipListNode<T>> *preds,
    std::shared_ptr<ConSkipListNode<T>> *succs) -> int {
    int layer = -1;
    std::shared_ptr<ConSkipListNode<T>> pred = LSentinel_;
    for (int i = this->max_layer_ - 1; i >= 0; --i) {
        std::shared_ptr<ConSkipListNode<T>> curr = pred->next_[i];
        // if we add RSentinel_ to the end of the list, curr is never nullptr
        while (curr != nullptr && curr->key_ < key) {
            pred = curr;
            curr = curr->next_[i];
        }
        if (layer == -1 && curr != nullptr && curr->key_ == key) {
            layer = i;
        }
        preds[i] = pred;
        succs[i] = curr;
    }
    return layer;
  }
  \end{lstlisting}
  \caption{FindNode函数}
  \label{fig:code1}
\end{figure*}


\subsection{并发跳表}
并发跳表的实现有很多种,其中最简单的一种是使用锁来保证并发安全,但是这种方法的性能并不好,因为在并发环境下,锁的开销很大。Herlihy提出了一种基于乐观锁的算法,该算法的核心思想是:在插入和删除时,不立即修改数据结构,而是在查找时
进行修改,如果发现数据结构被修改,则重新查找,直到成功为止。该算法的优点是实现简单,缺点是在高并发的情况下,会有大量的重试,导致性能下降。

\section{算法描述}
和普通的wait-free的链表算法相比,跳表的算法更加复杂,因为跳表的每个节点都有多个层级,所以相比单链表的并发插入,跳表只有等所有层级都被链入才被认为插入成功,该时刻也是linearization point。因此,跳表的并发插入算法需要保证
所有层级都被链入,否则需要回滚。Herlihy的算法使用了乐观锁,即在插入时不立即修改数据结构,而是在查找时进行修改,如果发现数据结构被修改,则重新查找,直到成功为止。

其次,跳表在插入之前的查找也很有讲究。如果像链表一样只记录一个前驱节点和后继节点,就只有该邻居的指针能被成功链入。因此,跳表的查找需要记录每一层的前驱节点和后继节点,这样才能保证所有层级都被成功链入。
查找时原论文使用使用辅助函数\verb|FindNode|(图\ref{fig:code1}),该函数使用\verb|preds|和\verb|succs|两个数组来记录每一层的前驱节点和后继节点,并且不做任何前驱后继的validation,从整个实验过程来看,
该辅助函数被频繁调用,因此不做额外检查可以提高性能。

\subsection{Add函数}
Add主要被分为三个过程:查找目标位置并验证,加锁并插入节点,解锁并返回。
使用FindNode查找插入位置,由于FindNode并不加锁,因此在函数返回后为了确保正确需要进行两步检查:
[1].待插入节点在跳表中是否存在;[2].如果存在,是否被mark。在确定未被mark并且存在后,该线程
需要等到key完全被链入跳表再返回false。

当key不在跳表中时,对preds和succs中暂存的每一层的前驱和后继节点,从下而上扫描到准备给新加入节点分配的高度值$h$,并依次给每一层的pred加锁。
或者找到最低的不满足条件的层,这里的条件是指该层的pred和succ都没有被mark且pred的后继的确是succ,即没有其他线程在pred和succ之间插入。

如果在对$h$之下的每一层检查都成立,该节点可以被安全的加入到跳表中,
链入后设置\verb|fully_linked|为true,否则解锁并从头开始尝试。

有关Add和Remove的具体实现见附录。

\subsection{Remove 函数}
Remove与Add类似,首先调用FindNode看看目标节点是否存在,然后依次检查以下条件是否成立:
[1].当前节点是否需要被删除,该判断为了检查上一次失败的删除操作是否有需要删除的节点;或者:[2].当前节点是否被完全链入跳表(\verb|fully_linked)|;
[3]. 节点的最高层是否和从上到下寻找的最高层一致;[4].节点是否还没被mark。

当上述条件满足时,节点的\verb|marked|被设置为true,并执行与\verb|Add|对preds和succs数组从下而上做validation同样的检查。
检查成功,每一层的pred直接连接到succ即可,\verb|shared_ptr|会自动回收被删除的节点。

\subsection{Contains}
contains直接调用\verb|FindNode|,并检查找到的节点是否未被marked并且是否完全链入跳表。

\section{算法正确性}
\subsection{Linearizability}
Linearizability的正确性由lazy list\cite{pugh_skip_1990}的正确性保证,即对于算法的每一个可达状态,节点的前驱和后继关系
在之后的任何过程中都能得到保证,直到某个节点被移除。在移除过程中,被物理移除的节点必须被标记为marked,并且由于\verb|fully_linked|
只会从\verb|false|标记为\verb|true|,于是一个节点只可以通过标记marked来移除,而设置marked的指令是linearization point。

对节点的加入操作只有在\verb|fully_linked|被标记为true时才算做完。在此之前,Contains和Remove即使
真的找到该节点,也不会认为节点被完全加入到跳表,因为新节点的链接关系还未被完全设置。于是\verb|fully_linked|也是linearization point。

最后需要说明的是,某个节点的\verb|marked|在程序执行的某个时刻被发现标记为true,那么该node的key一定不在跳表中。
一方面,被标记的节点在检查之前一定在跳表中,因为:[1].只有\verb|marked|为false的节点才能被加入list;[2].key在跳表中唯一,
不可能有unmarked的节点具有相同的key。另一方面,如果看到节点被标记,说明在该线程之前一定有某个线程正在负责该节点的删除操作,
只不过还未把节点从当前层摘下从而被当前线程发现。对于这种情况,当负责摘除的线程完成删除之后,只有当前线程的中的某个临时指针指向
该节点,临时指针释放后,该节点被自动物理释放。

\subsection{不变性保证}
在加入和删除时之前都会对前驱和后继的有效性做检查,检查的同时会给每一层加锁。加入时从上到下将节点链入,删除时从下到上将节点摘除。
假设在某一时刻两个线程希望同时加入一个相同的key,在validation时,两个线程在竞争第0层的锁时只有一个线程能够成功,而另一线程
在第0层锁上等待。等到竞争成功的线程完全持有每一层的锁并物理加入节点后,等待的线程被唤醒并发现记录的前驱和后继不匹配,于是重新
\verb|FindNode|,发现节点已经被加入,返回false。

删除节点时同理。

对于朴素跳表算法,由于不加锁,在两个线程加入相同key的节点时,最好的情况是可能导致具有重复key的节点被加入,删除时只删除其中一个,破坏了不变性。最坏的情况是
每一层的前驱指针分别随机指向具有相同key的不同节点,破坏了链表的结构。

\subsection{deadlock-free \& wait-free}
由于\verb|FindNode|按照从小到大的顺序查找目标位置,算法总是优先操作较小的key,较大key在较低层等待。因此能够保证到来的每个key都能按顺序被操作。
不会出现死锁。同理Contains操作也是wait-free的,首先他不持有任何锁,并且只对跳表做一次扫描。

\section{实验结果}
本次实验机器的CPU是Apple M1 (8 cores, 8 threads),测试一执行200,000条Add指令,测试二总共执行200,000条指令操作,其中90\% Add指令,
7.5\%的Remove指令以及2.5\%的Contains指令。

% Number of threads: 1
% 200000 / 200000
% Time elapsed: 304560ms
% Number of threads: 2
% 200000 / 200000
% Time elapsed: 196218ms
% Number of threads: 4
% 200000 / 200000
% Time elapsed: 115205ms
% Number of threads: 8
% 200000 / 200000
% Time elapsed: 96267ms

% Number of threads: 1
% 200000 / 200000
% Time elapsed: 279979ms
% Number of threads: 2
% 200000 / 200000
% Time elapsed: 167576ms
% Number of threads: 4
% 200000 / 200000
% Time elapsed: 94557ms
% Number of threads: 8
% 200000 / 200000
% Time elapsed: 79929ms

% 测试1和测试2的表格
\begin{table}[ht]
  \centering
  \caption{测试1:200,000条Add指令}
  \begin{tabular}{cccc}
    \toprule
    线程数 & 指令数 & 时间(ms) & 每秒指令数 \\
    \midrule
    1 & 200,000 & 304,560 & 656 \\
    2 & 200,000 & 196,218 & 1,019 \\
    4 & 200,000 & 115,205 & 1,735 \\
    8 & 200,000 & 96,267 & 2,076 \\
    \bottomrule
  \end{tabular}
  \label{tab:table1}
\end{table}

\begin{table}[ht]
  \centering
  \caption{测试2:200,000条指令,其中90\% Add指令,7.5\%的Remove指令以及2.5\%的Contains指令}
  \begin{tabular}{cccc}
    \toprule
    线程数 & 指令数 & 时间(ms) & 每秒指令数 \\
    \midrule
    1 & 200,000 & 279,979 & 714 \\
    2 & 200,000 & 167,576 & 1,193 \\
    4 & 200,000 & 94,557 & 2,115 \\
    8 & 200,000 & 79,929 & 2,501 \\
    \bottomrule
  \end{tabular}
  \label{tab:table2}
\end{table}

实验结果如表\ref{tab:table1}和表\ref{tab:table2}所示,可以看到随着线程数的增加,每秒指令数也随之增加,但是增加的幅度逐渐减小。由于生成Remove和Contains指令
的key在0-200,000的\verb|int|型数据中采样,所以执行中可能由于节点未被加入而提前返回,并且contains只包含简单的查找与谓词逻辑,因此吞吐量会稍高于测试一。

\section{讨论}
\subsection{模糊搜索HNSW算法}
HNSW(Hierarchical Navigable Small World)\cite{malkov_efficient_2020}是一种基于图的模糊搜索算法,其核心思想是:在高维空间中,每个节点都有一些邻居,这些邻居的距离都比较近,但是不一定是最近的。这样的邻居被称为“小世界”。
该索引已经作为向量数据库的基础索引被广泛使用,其性能已经超过了sota的KD-Tree算法。HNSW索引第0层是一个关于数据集邻接关系的图,以上每一层都是下一层的子图。对于一个向量搜索请求,从高层向下搜索,与跳表不同的是,由于采用基于最近邻的
模糊搜索,HNSW需要从最高层一直找到最底层。

和跳表不同的是,HNSW对Remove的支持并不友好,图中节点的删除可能会导致图不连通,因此只能重新构建索引。
因此我们在此仅讨论针对HNSW的Add和Contains的并发化的一些可能思路。

在插入节点之前,节点被赋予\verb|top_layer|(与跳表含义相同)。搜索时采用不加锁的贪心搜索,从当前层的起点开始找到k个最近邻(k为当前层的超参数)。
如果当前层需要与邻居节点加边($layer \leq top\_layer$),那么将所有邻居加锁,并把当前节点加入到邻居的邻居列表中,解锁并返回,这里可能需要检查
邻居节点的邻居数是否与进入临界区前记录的邻居数是否相同,防止在相邻位置有新加入的节点。如果有新加入的节点,可能需要重新执行当前层的搜索(不需要从最高层开始查找)。

Contains操作被重新定义为找到当前查询的k个最近邻,可以不加锁地从高到低搜索。

\subsection{跳表的优化}
\verb|FindNode|函数被三个函数都会被调用一次,因此该操作具有很大的优化空间。比如可以通过记录跳表当前具有的最高层数,从而不必每次从最高层开始扫描,由于每一层节点数相比下一层
指数级降低,因此并没有很多节点能够拥有跳表允许的最高层,当层数增加时该现象更加明显。另外,可以通过记录每一层的最后一个节点,从而在查找时可以从上一次查找的位置开始,而不必从头开始扫描。

\section{总结}
本次课程设计复现了Herlihy的并发跳表算法,并对算法的正确性进行了分析。实验结果表明该算法在高并发环境下有很好的性能提升,且实现简单,易于调试与维护。最后,讨论了HNSW算法的并发化可能的思路,总结了跳表的一些优化空间。

% ieee style
\bibliographystyle{IEEEtran}
\bibliography{paper}

\onecolumn
\appendix
\section{附录}
\subsection{Add函数}
\begin{lstlisting}[language=c++]
template<typename T>
auto ConSkipList<T>::Add(T key) -> bool {
  std::shared_ptr<ConSkipListNode<T>> preds[this->max_layer_];
  std::shared_ptr<ConSkipListNode<T>> succs[this->max_layer_];
  while (true) {
      int layer_check = FindNode(key, preds, succs);
      if (layer_check != -1) {
          std::shared_ptr<ConSkipListNode<T>> nodeFound = succs[layer_check];
          if (!nodeFound->marked_) {
              // wait until node is fully linked, i.e. node is visible.
              while (!nodeFound->fully_linked_) {}
              return false;
          }
          continue;
      }
      int highestLocked = -1;
      int top_layer = this->RandomLayer();
      std::shared_ptr<ConSkipListNode<T>> pred, succ, prevPred = nullptr;
      bool valid = true;
      for (int layer = 0; valid && layer <= top_layer; ++layer) {
          pred = preds[layer];
          succ = succs[layer];
          // check if pred and succ are still valid
          if (pred != prevPred) {
              pred->lock_.lock();
              highestLocked = layer;
              prevPred = pred;
          }
          valid = !pred->marked_ && !succ->marked_ && pred->next_[layer] == succ;
      }
      if (!valid) {
          for (int layer = 0; layer <= highestLocked; ++layer) {
              preds[layer]->lock_.unlock();
          }
          continue;
      }
      std::shared_ptr<ConSkipListNode<T>> newNode = std::make_shared<ConSkipListNode<T>>(key, top_layer);
      for (int layer = 0; layer <= newNode->top_layer_; ++layer) {
          newNode->next_[layer] = succs[layer];
          preds[layer]->next_[layer] = newNode;
      }
      // linearization point
      newNode->fully_linked_ = true;
      for (int layer = 0; layer <= highestLocked; ++layer) {
          preds[layer]->lock_.unlock();
      }
      return true;
  }
}
\end{lstlisting}

\subsection{Remove函数}
\begin{lstlisting}[language=c++]
template<typename T>
auto ConSkipList<T>::Remove(T key) -> bool {
    std::shared_ptr<ConSkipListNode<T>> victim = nullptr;
    bool isMarked = false;
    std::shared_ptr<ConSkipListNode<T>> preds[this->max_layer_];
    std::shared_ptr<ConSkipListNode<T>> succs[this->max_layer_];
    while (true) {
        int layer_check = FindNode(key, preds, succs);
        if (layer_check != -1) {
            victim = succs[layer_check];
        }
        if (isMarked ||
            (layer_check != -1 &&
              victim->fully_linked_ && victim->top_layer_ == layer_check && !victim->marked_)) {
            if (!isMarked) {
                victim->lock_.lock();
                if (victim->marked_) {
                    victim->lock_.unlock();
                    return false;
                }
                victim->marked_ = true;
                isMarked = true;
            }
            int highestLocked = -1;
            std::shared_ptr<ConSkipListNode<T>> pred, succ, prevPred = nullptr;
            bool valid = true;
            for (int layer = 0; valid && layer <= victim->top_layer_; ++layer) {
                pred = preds[layer];
                succ = succs[layer];
                if (pred != prevPred) {
                    pred->lock_.lock();
                    highestLocked = layer;
                    prevPred = pred;
                }
                valid = !pred->marked_ && pred->next_[layer] == succ;
            }
            if (!valid) {
                for (int layer = 0; layer <= highestLocked; ++layer) {
                    preds[layer]->lock_.unlock();
                }
                continue;
            }
            for (int layer = victim->top_layer_; layer >= 0; --layer) {
                preds[layer]->next_[layer] = victim->next_[layer];
            }
            victim->lock_.unlock();
            for (int layer = 0; layer <= highestLocked; ++layer) {
                preds[layer]->lock_.unlock();
            }
            return true;
        } else {
            return false;
        }
    }
}
\end{lstlisting}

\subsection{Contains函数}
\begin{lstlisting}[language=c++]
template<typename T>
auto ConSkipList<T>::Contains(T key) -> bool {
    std::shared_ptr<ConSkipListNode<T>> preds[this->max_layer_];
    std::shared_ptr<ConSkipListNode<T>> succs[this->max_layer_];
    int layer = FindNode(key, preds, succs);
    return (layer != -1 && succs[layer]->fully_linked_ && !succs[layer]->marked_);
}
\end{lstlisting}

代码与

\end{document}
